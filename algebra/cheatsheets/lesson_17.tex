\documentclass[a4paper]{article}
\usepackage[utf8]{inputenc}
\usepackage[T2A]{fontenc}
\usepackage[english]{babel} 
\usepackage[left=20mm, top=20mm, right=20mm, bottom=20mm, nohead, includefoot]{geometry} 
\usepackage{amsmath,amsfonts,amssymb} % математический пакет
\usepackage{fancybox,fancyhdr} 
\usepackage{tikz}
\pagestyle{fancy}
\headsep=5mm 
\setlength{\footskip}{12.0pt}
\setlength{\headheight}{12.0pt}
\newcommand{\lr}[1]{\left({#1}\right)} % команда для скобок

\fancyfoot[C]{Made by Egor Kharitonov with \LaTeX{}} 
\fancyhead[L]{{\large \textbf{Math for Kristina. CheatSheets. Lesson 17.}}}
\fancyhead[R]{{\large \textbf{Summer 2022}}}

\begin{document}

\begin{Large}

\section*{Power of a number}

\subsection*{Definition}

$a^n = \smash[b]{\! \underbrace{a \cdot a \cdots{a}\,}_\text{$n$ times}}$

\subsubsection*{Example:}

$2^4 = 2 \cdot 2 \cdot 2 \cdot 2 = 16$ \\[0.05cm]

\noindent $5^3 = 5 \cdot 5 \cdot 5 = 125$ \\[0.05cm]

\noindent $10^5 = 100000$

\subsection*{Attributes}

\subsubsection*{Product of powers}

$a^m \cdot a^n = a^{m+n}$ \\[0.05cm]

\noindent $3^4 \cdot 3^2 = 3^6 = 729$

\subsubsection*{Division of powers}

$a^m \div a^n = \dfrac{a^m}{a^n} = a^{m-n}$ \\[0.05cm]

\noindent $3^4 \div 3^2 = \dfrac{3^4}{3^2} = 3^2 = 9$

\subsubsection*{Multiplication power of power}

$(a^m)^n = a^{m \cdot n}$ \\[0.05cm]

\noindent $(2^2)^3 = 2^{2 \cdot 3} = 2^6 = 64$

\subsubsection*{Power of fraction}

$(a^n \cdot b^n) = (a \cdot b)^n$ \\[0.05cm]

\noindent $(a^n \div b^n) = (a \div b)^n$

\section*{Dot, lines, ray}

\begin{tikzpicture}

    \filldraw[black] (0,0) circle (2pt) node[anchor=west]{};
    \filldraw[black] (4.5,0) circle (1pt) node[anchor=west]{Dot};

\end{tikzpicture}
\\[0.1cm]

\begin{tikzpicture}
    \draw[gray, thick] (-1,2) -- (2,2);
    \filldraw[black] (-1,2) circle (2pt) node[anchor=west]{};
    \filldraw[black] (3,2) circle (1pt) node[anchor=west]{Ray};
    \\[0.1cm]

    \draw[gray, thick] (-1,0) -- (2,0);
    \filldraw[black] (3,0) circle (1pt) node[anchor=west]{Line};

\end{tikzpicture}

Ray can extend only in one way endlessy. Line can be extend in two ways endlessly. Dot is a dot.

\end{Large}

\end{document}